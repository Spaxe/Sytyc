\documentclass[a4paper]{article}
\usepackage{graphicx}
\usepackage{hyperref}
\usepackage{fancyvrb}
\usepackage{listings}
\usepackage{color}
\usepackage{textcomp}
\DefineShortVerb{\|}

\hypersetup{
    colorlinks=true,      
    urlcolor=red      
}

\lstset{
  tabsize=2,
  rulecolor=,
  language=haskell,
  basicstyle=\normalsize,
  upquote=true,
  aboveskip={1.5\baselineskip},
  columns=fixed,
  showstringspaces=false,
  extendedchars=true,
  breaklines=true,
  prebreak = \raisebox{0ex}[0ex][0ex]{\ensuremath{}},
  showtabs=false,
  showspaces=false,
  showstringspaces=false,
  identifierstyle=\ttfamily,
  keywordstyle=\color[rgb]{0,0,1},
  commentstyle=\color[rgb]{0.133,0.545,0.133},
  stringstyle=\color[rgb]{0.627,0.126,0.941},
}

\title{Sytyc-1.0}
\author{Xavier Ho,\quad s2674674}
\date{\today}

\begin{document}
\maketitle

\section*{\centering \small Abstract}

\href{https://github.com/SpaXe/Sytyc}{Sytyc} is a web-based programming puzzles database with a compiler on the server side.  The web interface automatically parses the system sub-folders and generates a list of problems end-users can pick from.  The end-user solves the problem and submits some source code.  Then, Sytyc compiles and runs the code on the server, and checks if the program passes all tests.  Errors are reported directly on the webpage.  Successful programs will past all tests, and the end-user is congratulated.

\section{Background}

Originally started as "Judge", Sytyc was conceptualised to provide students an independent platform where they can test their programs, while solving puzzles with difficulties ranging from introductory to very difficult.  It also serves as a platform to train ICPC programming competition participants: format of the puzzles has been bulit similar to the competition format.  Sytyc archives puzzles by the subfolders in the file system, which is easy to backup and maintain.

There are existing websites that have done an excellent job, such as \href{http://www.topcoder.com/}{TopCoder} and \href{http://projecteuler.net/}{ProjectEuler}.  Sytyc is an attempt to recreate some features from these sites in order to customise the types of problems in the database, appropariate for the target programmers at Griffith University.  In addition, Sytyc supports Andrew Rock's \href{http://www.ict.griffith.edu.au/arock/MaSH/index.html}{MaSH} compiler, which is consistent with the offering with the introductory programming course on Nathan campus.

\section{Installation}

This installation guide assumes the reader is familiar with basic Unix commands and its operating system envorinment.

\subsection{Checking out the source code}

If, for some reason, the source code was not provided along with the manual, you can always obtain the latest version of Sytyc at its  \href{https://github.com/SpaXe/Sytyc}{Github repository}.

If you're not a Git user, download and install \href{http://git-scm.com/}{Git}.  Open up a terminal and check out the soure code with:
\begin{verbatim}
git clone git@github.com:SpaXe/Sytyc.git
\end{verbatim}

\subsection{Compiling the source code}

Sytyc requires the following dependencies to be installed:
\begin{verbatim}
GHC 7.0.3       pandoc-1.8      missingH-1.1.0.3
\end{verbatim}
in addition to the \href{http://hackage.haskell.org/platform/}{Haskell Platform} (2011.2.0.1).  Sytyc has been tested on Windows 7 with XAMPP 1.7.4, compiled against the Haskell Platform of the version above.  The server must support CGI for Sytyc to function.

Move the Sytyc source code to a CGI-enabled folder.  The index page should be index.cgi.

If you're on Windows, in the main folder of Sytyc simply type
\begin{verbatim}
make
\end{verbatim}

If you're on Mac OS X or Linux, in the main folder of Sytyc type
\begin{verbatim}
cd src
make
make install
cd ..
\end{verbatim}

You should see several |.cgi| files being compiled and moved to the main folder.  Open up a web browser and check that you can see the main webpage.  Ensure your CGI environment permission is set properly so that connections cannot see your subdirectories for basic security.

\section{Using Sytyc}
The majority of Sytyc should be self-explanatory as you explore the webpage.  This version of Sytyc supports only MaSH source code with the \href{http://www.ict.griffith.edu.au/arock/MaSH/console.html}{console} environment.  You are encouraged to try the sample problems and see Sytyc in action.

If you encounter any issues, please contact Xavier at \href{mailto:contact@xavierho.com}{contact@xavierho.com}.

\section{Customising Sytyc}
Thanks to the \href{http://daringfireball.net/projects/markdown/}{Markdown} syntax, the majority of the webpage is generated by plaintext templates found in the |templates| folder.  There are two types of templates: HTML files, and |.md| files.

\subsection{HTML Files}

|template.html| contains the main structure of the entire webpage, from the |DOCTYPE| all the way to the end. You will see certain template variables surrounded by the |$| sign.  These variables are substituted with the appropriate content at run-time.

|problems.html| (plural) is a partial webpage that contains the HTML structure for a list of problems.

|problem.html| (singular) is also a partial webpage that contains the HTML structure for displaying a problem description, an example of input and output, as well as the textarea in which end-users can submit their source code.

\subsection{Markdown Files}

\href{http://daringfireball.net/projects/markdown/syntax}{Here is the complete reference of the markdown syntax}.  Generally speaking, the markdown files are very easy to understand and modify.  An example looks like:
\begin{verbatim}
Sytyc - Programming Puzzles
===========================
If you're here, you probably can program in at least one language.
Perhaps you wanted to learn a new language, or perhaps you just
wanted to tackle some programming challenges.

Whatever the reason is, welcome.

Get Started
-----------
If you're a first-timer,
[check out the list of problems here](problem_list.cgi). Get your 
compiler or interpreter and your favourite IDE ready, and start
coding.  All problems use the standard output for printing, and
standard input for reading.

Got questions?
--------------
There is a list of [Frequently Asked Questions](faq.cgi). If your
question isn't on the list, you can [give us a shout](feedback.cgi).

What is Sytyc?
--------------
Sytyc is an interface where you can view programming puzzles, and
submit your answers in source code.  It's entirely written in
Haskell, and runs via CGI on any web browser.  It's also a great
place to learn programming.
\end{verbatim}

Note that hyperlinks is simply |[display-name](link)|. '|=|' denotes a |h1| in HTML-speak, while a '|-|' denotes a h2.  You don't have to recompile Sytyc if you modify these |.md| templates---they are parsed at run-time.

\end{document}